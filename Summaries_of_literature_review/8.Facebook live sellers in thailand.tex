\documentclass[a4paper]{article}
\usepackage[utf8]{inputenc}
\usepackage{authblk}
\usepackage[margin=0.5in]{geometry}


	
	

\title{Facebook live sellers in thailand}
\author[1]{Nassim Dehouche, Deval Shaileshkumar Mali}


\begin{document}
\maketitle
\section{The main idea}

This paper was published to serve as a basis for research on customer engagement with one of the novel sales channel that is Facebook Live, through comparative studies with other forms of content such as text, deferred videos, and images, as well as the statistical analysis of the seasonality of engagement and outlier posts.

\section{The methodology }

The dataset consists of 7050 Facebook posts of various types as text, deferred and live videos, images. These posts were extracted from the Facebook pages of 10 Thai fashion and cosmetics retail sellers from March 2012 to June 2018. The dataset was collected via the Facebook API and anonymized in compliance with the Facebook Platform Policy for Developers. 

The variability of consumer engagement is analyzed through a Principal Component Analysis, highlighting the changes induced by the use of Facebook Live. The seasonal component is analyzed through a study of the averages of the different engagement metrics for different time-frames (hourly, daily and monthly). Finally, we identify statistical outlier posts, that are qualitatively analyzed further, in terms of their selling approach and activities.

First the descriptive statistics of engagement metrics for each Facebook seller in the Dataset was measured and presented in tabular form data. That table presents the mean, standard deviation, and maximum value of the considered engagement metrics.

Facebook pages were selected based on their number of followers and activity, using the Facebook Live Map tools. Data were collected through a Python script that makes queries to the Facebook API using the URL pattern, in which “Pagename” is the name of the page of the seller, “StartDate” and “EndDate”, respectively the date of the first ever post by a page and the current date and “Token”, our access token to the Facebook API. 

\section{The results}
 
The conclusion is that as vividness increases, while interactivity decreases the level of engagement over moderator posts, and photos are the most appealing post media type. If you provide entertaining and informative content that would significantly increases the level of engagement. Using social posts is likely to elicit comments and encourage the interaction of users.


\section{Recommendation}

Researchers could have used more visualizations by utilizing python matplotlib library or visualization tools such as tableau, SAS (Statistical Analysis System). They could have analyzed more data using statistical functions/libraries in python so user get more accurate results for model and also for the comparison.

\end{document}