\documentclass[a4paper]{article}
\usepackage[utf8]{inputenc}
\usepackage{authblk}
\usepackage[margin=0.5in]{geometry}


	
	

\title{Customer Segmentation using Centroid Based and Density Based Clustering Algorithms}
\author[1]{A.S.M Shahadat Hossain, Tirumalesh Reddy Neelapu}


\begin{document}
\maketitle
\section{The main idea}

As per Smith, Market or Customer segmentation involves viewing a heterogeneous market as several smaller homogeneous markets in response to differing preferences, attributable to the desires of consumers for more precise satisfaction of their varying wants. Ensuring customer satisfaction and for optimal profits, customer segmentation helps understanding different behaviors of customer. 

\section{The methodology }

This paper illustrates the idea of applying density-based algorithms for customer segmentation beside using centroid based algorithms like k-means. Also applied DBSCAN (Density-Based spatial Clustering of applications with Noise) algorithm as one of the density-based algorithms results in a meaningful customer segmentation. This paper uses the Wholesale customers dataset from University of California and refers to clients of a wholesale distributor and contains data regarding consumption of different items by customers showing their annual spending.

Centroid Based Clustering, a predefined number of points are selected randomly which are called ‘Centroid’ where that predefined number denotes the number of expected clusters.
The results

Density Based Clustering discovers clusters with arbitrary shape based on the neighborhood principle which means that points within radius are neighbors.


\section{The results}
 
Researchers presented a comparison between performances of k-means and DBSCAN for customer segmentation. From the comparison, we can infer that DBSCAN takes relatively longer time than k-means. 

\section{Recommendation}

Paper is restricted to only density based and k-means clustering algorithms and the metrics are compared against the same. Working with other clustering algorithms would have provided better comparison metrics and proposing the best clustering algorithm for the chosen dataset.

\end{document}