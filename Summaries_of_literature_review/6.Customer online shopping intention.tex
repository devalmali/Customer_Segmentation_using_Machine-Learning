\documentclass[a4paper]{article}
\usepackage[utf8]{inputenc}
\usepackage{authblk}
\usepackage[margin=0.5in]{geometry}


	
	

\title{How e-services capes affect customer online shopping intention: the moderating effects of gender and online purchasing experience}
\author[1]{Wann-Yih Wu, Phan Thi Phu Quyen, Deval Shaileshkumar Mali}


\begin{document}
\maketitle
\section{The main idea}

E-commerce is expanding at a tremendous rate, and the related internet pages, referred to as an electronic services cape (e-services capes) now have a significant presence in the business world. Nevertheless, understanding of e-services cape attributes remains unclear, due to the limited empirical evidence that has been obtained and examined. This study aims to describe the nature of e-services capes and investigate the relationships among website trustworthiness, website attitude, brand attitude, e-WOM intention, and purchase intention. Furthermore, this study also aims to identify the role of two contextual factors, namely online purchasing experience and gender differences, and their effects on the relationships among the e-services cape dimensions, website trustworthiness and attitude. A total of 290 responses were collected from Taiwanese consumers using online-based questionnaires. SPSS and partial least square were used to analyze the collected data.  

\section{The methodology }

The dataset consists of feature vectors belonging to 12,330 sessions. The dataset was formed so that each session would belong to a different user in a 1-year period to avoid any tendency to a specific campaign, special day, user profile, or period.  

A research model is developed based on an integration of the above research concepts and the S–O–R framework, TPB, and DMH. Steel and Ko¨nig (2006) argued that synthesizing different theories into a comprehensive model may be beneficial for the following reasons: (1) a single theory is not able to explain the whole phenomena; (2) it can deal with more realistic situations; (3) it is possible to approach problems in a more effective manner by improving the levels of complexity, and thus new insights and bring solutions can be obtained which cannot be reached through the use of a single theory. 

The study draws on S–O–R theory to create a causal chain, integrating the stimulus (e-servicescape), organism (trust in website, attitude toward website and attitude toward brand), and final outcome (e-WOM intention and purchase intention). The research model shows that the three dimensions of e-servicescape work as a stimulus transformed by attitudes, states and processes called an organism, which in turn creates a response. This study underpins its rationale regarding the positive relationships among website trustworthiness, website attitude and brand attitude with the TPB, which asserts the importance of paying attention to external factors in order to explain the behavioral intentions of consumers. Furthermore, based on the DMH, which is a widely used model for explaining the influence of advertising attitudes on consumer purchase intentions , this study employs the extended dual mediation hypothesis, which creates an analogy between advertising attitudes and website attitudes. 

In the first stage of the data collection process, 10 fashion brand websites in Taiwan were examined using convenience sampling of 35 graduate and undergraduate students. The second stage consisted of a pre-test that verified the reliability and validity of the measurement scales, based on replies from 100 respondents , the participants then completed the survey.


\section{The results}
 
A total of 290 valid questionnaires were collected. More than 61 \% of the respondents were female, and more than 58 \% of them were aged between 20 and 24 years old. In addition, more than 80 \% of the respondents had a bachelor’s degree or above, and more than 60 \% had a low level of experience of online purchasing. Among the respondents, 67.6 \% reported shopping online between one to three times within the last year. Finally, the effects that gender has on the relationships between the e-servicescape dimensions and trust and attitude toward a website suggests that website designers should put more effort into the appearance of sites if they are aimed at women, as is the case with online stores that sell cosmetics, clothes, or health products. The findings also show that having high levels of shopping experience will positively impact consumer attitudes and beliefs toward a website. 

Finally, prior studies have explained that culture plays a significant role in an online context, and thus further studies carried out in other countries could deepen our understanding of the moderating role of culture.

\section{Recommendation}

In future research they could include comparison and analysis with more websites, as well as adding trustworthiness and attitude as variables. Moreover, future studies may validate the moderator roles of other factors, such as customer characteristics (e.g., age and education) and customer involvement, particularly regarding the potential influence of e-servicescape on website trustworthiness and attitude.

\end{document}