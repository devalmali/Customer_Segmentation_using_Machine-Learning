\documentclass[a4paper]{article}
\usepackage[utf8]{inputenc}
\usepackage{authblk}
\usepackage[margin=0.5in]{geometry}


	
	

\title{RFM ranking - An effective approach to customer segmentation}
\author[1]{A. Joy Christy, A. Umamakeswari, L. Priyatharsini, A. Neyaa, Abhishek Puppala}


\begin{document}
\maketitle
\section{The main idea}

The main idea of the paper is to separate the customers into segments to increase the total revenue of the organization or company. They believe that keeping back the customer is way more important than discovering new customers. So, they decided to deploy marketing strategies that are tailored to a certain market niche in order to keep customers. So, the researchers decided to use RFM analysis and K-means, Fuzzy-C clustering methods to obtain the required results. The main aim of this research is to selecting the centroids for k-means technique and to reduce iteration and time when segmenting the customers. 

\section{The methodology }

In this study, the transactional dataset which contains the customer’s purchase information which contains a total of 18,267 data records with eight attributes such as stock code, description, Quantity, Invoice number, Invoice date, Custom Id, Unit Price, and Country. The dataset is obtained from the repository of the obtained from the University of California, Irwin. Recency, Frequency, and Monetary (RFM) analysis is a popular method for evaluating customers based on their buying habits. A points system is designed to evaluate the Recency, Frequency, and Monetary scores. The RFM score, which varies from 555 to 111 and is utilized to predict future behaviors by evaluating past and current customer experiences, is the outcome of combining the results from all three factors. In this scenario, it has been discovered that customer duration and maintenance are directly correlated with the scores of the three factors. The factors used to generate customer base groups are exposed to the K-Means algorithm. The customer group that produces the most income for the company is determined by examining the activities of each cluster.

\section{The results}
 
The system time is used to compute each algorithm's execution time. Due to the fewer repetitions, it has been found that the suggested Repetitive Median (RM) K- Means technique takes less time than the other two procedures. Because the initial centroids are determined using median values, the RM K-Means requires less iterations. The corporation may customize its advertising strategies to the individuals depending on their buying behaviour because segmentation is performed based on the principles of recency, frequency, and monetary values. 

\section{Recommendation}

Future research can look into customer behavior in each category, such as the commodities that customers in each component purchase frequently. This might facilitate the ability to provide particular products effective promotional opportunities. 

\end{document}